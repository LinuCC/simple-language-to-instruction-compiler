\part{Theorie zum Compiler}
\chapter{Compiler Uebersicht}
\label{chapter}

\begin{itemize}
  \item Compiler Historie
  \item Einleitung zu den detaillierteren Beschreibungen:
  \item Arten von Compilern 
  \begin{itemize}
  \item[->] Spaeter genauer erklaert: Top-Down Compiler
  \end{itemize}
  \item Compilerphasen
  \begin{itemize}
    \item Einpass vs Mehrpass vs Zweipass zur Trennung Frontend <-> Backend
    \item[->] Spaeter genauer erklaert: Zweipass Compiler fuer separate Backends & zukuenftige Optimierungsschritte
  \end{itemize}
  \item Parser: Arten (EBNF etc)
  \begin{itemize}
  \item[->] Spaeter genauer erklaert: EBNF
  \end{itemize}
  \item Zwischensprache: Vereinfacht Optimierungen, separierte Backends und Frontends
  \begin{itemize}
    \item[->] Spaeter genauer erklaert: ??? (Wie so ein Zwischenstate aussehen soll ist scheinbar schwarze Magie, selber rausfinden)
  \end{itemize}
  \item Backends: Code Erzeugung etc
  \begin{itemize}
    \item[->] Spaeter genauer erklaert: RISC Binary, RISC assembly
  \end{itemize}
  \item Bootstrapping compiler in eigener Sprache
  \begin{itemize}
    \item[->] Passiert hier nicht, zu komplex und wild
  \end{itemize}
\end{itemize}
