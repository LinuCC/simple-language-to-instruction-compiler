\chapter{Der Compiler im Zweipass-Verfahren}
\label{chap:theory:zweipassCompiler}

Historisch wurde der Mehrpass-Compiler dem Einpass-Compiler vorgezogen, da das Unterteilen der Einzelschritte in mehrere Programmteilen die Nutzung von Arbeitsspeicher verringerte und das Compilieren von komplexen Programmiersprachen ermöglichte.
Mit den Verbesserungen an Hard- und Software ist beides kein Problem mehr.
Deswegen werden dem Mehrpass-Compiler entweder der Einpass-Compiler oder eine spezielle Variante des Mehrpass-Compilers, der Zweipass-Compiler, heutzutage bevorzugt \cite{mossenbock:2024}.\\
Im Zweipass-Compiler werden die Phasen des Compilers in zwei Programmteile gruppiert.
Der erste Teil, das Frontend des Compilers, überprüft den Programmcode auf Syntax und Semantik;
Aus diesen Informationen wird eine Zwischensprache generiert, die den Programmcode in einer internen Darstellung abbildet.
Der zweite Teil, das Backend, liest die Zwischensprache ein und generiert daraus Maschinencode.\\

\begin{figure}[H]
  \begin{tikzpicture}
  \sffamily
    \node[minimum height=1.5cm,minimum width=2cm,text centered,draw](cppfrontend){C++ Frontend};
    \node[above=1cm of cppfrontend, minimum height=1.5cm,minimum width=2cm,text centered,draw] (pascalfrontend){Pascal Frontend};
    \node[right=2cm of {$(cppfrontend)!0.5!(pascalfrontend)+(0,0.25)$}, minimum height=1.5cm,minimum width=2cm,text centered,draw] (state){Zwischensprache};
    \node[title,left=0.75cm of pascalfrontend, text width=2.5cm](pascalcode){Pascal\\
    Programmcode};
    \node[title,left=0.75cm of cppfrontend, text width=2.5cm](cppcode){C++\\
    Programmcode};
    \node[title,above=0.33cm of pascalfrontend](frontendstitle){Frontends};
    \node[above=1cm of pascalfrontend, fit={(pascalfrontend)($(cppfrontend.south)+(0,-1)$)},minimum width=2cm,draw=black!50] (frontendsbox);
    \node[right=4cm of pascalfrontend, minimum height=1.5cm,minimum width=2.5cm,text centered,draw](riscbackend){RISC Backend};
    \node[below right=1cm and 4cm of pascalfrontend, minimum height=1.5cm,minimum width=2.5cm,text centered,draw] (x86backend){X86 Backend};
    \node [title,above=0.33cm of riscbackend](backendstitle){Backends};
    \node[above=1cm of riscbackend, fit={(riscbackend)($(x86backend.south)+(0,-1)$)},minimum width=2cm,draw=black!50] (backendsbox);
    \node [title,right=0.75cm of riscbackend, text width=2cm](riscbinary){RISC\\
    Maschinencode};
    \node [title,right=0.75cm of x86backend, text width=2cm](x86binary){X86\\
    Maschinencode};
    \node[above=1cm of pascalfrontend, fit={(frontendsbox)($(backendsbox.southeast)+(0, -4.25)$)},minimum width=2cm,draw=black!50] (compilerbox);
    \node [title,above=2.5cm of state](compilertitle){Compiler};
    \umltrans[arg=]{pascalcode}{pascalfrontend}
    \umltrans[arg=]{cppcode}{cppfrontend}
    \umltrans[arg=]{riscbackend}{riscbinary}
    \umltrans[arg=]{x86backend}{x86binary}
    \umlVHtrans[arg=]{cppfrontend}{state}
    \umlVHtrans[arg=]{pascalfrontend}{state}
    \umlHVtrans[arg=]{state}{riscbackend}
    \umlHVtrans[arg=]{state}{x86backend}
  \end{tikzpicture}
  \caption{Flexibler Zweipass-Compiler mit mehreren möglichen Frontends und Backends}
\end{figure}

Dieser Aufbau bietet einige Vorteile gegenüber dem Einpass-Compiler:\\
Zum einen können bei gleicher Zwischensprache mehrere Varianten von Frontends und Backends gebaut und benutzt werden;
So kann man zum Beispiel eine andere Variante des Backends schreiben, um eine Programmiersprache in zwei unterschiedliche Prozessorarchitekturen zu übersetzen, ohne das Parsen nochmals zu schreiben.
Das Gleiche gilt auch für die einzulesenden Programmiersprachen, die mit unterschiedlichen Frontends geparst und mit dem gleichen Backend zu Maschinencode verarbeitet werden.\\
Zum anderen ist die Zwischensprache ein guter Punkt, um Optimierungen an der Struktur, Speichernutzung und so weiter vorzunehmen.
Diese generellen Optimierungen können dann noch von den Backends um Optimierungen, die für die Prozessorarchitektur spezifisch sind, erweitert werden.\\
Da die Optimierung von Programmcode ein wichtiger Teil moderner Compilern ist, sind diese überwiegend Zweipass-Compiler \cite{mossenbock:2024}.\\

Der in dieser Arbeit behandelte Compiler \ac{SLICC} verwendet das Zweipass-Verfahren, um mehrere Ausgaben zu unterstützen und zukünftige Optimierungen zu ermöglichen.
