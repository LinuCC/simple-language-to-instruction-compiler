\chapter{Der Parser}
\label{chap:theory:parser}

\section{Historisches / Details}

Einige Arten haben sich etabliert, um den Parser eines Compilers zu implementieren.
Die häufigsten davon sind Varianten des Top-Down und Bottom-Up Parsers, bei denen die Baumstruktur des Syntaxbaums entweder aus der Wurzel der Syntaxdefinition nach ``unten'' hin oder den einzelnen Blättern nach ``oben'' hin gebaut wird.\\
So gibt es zum Beispiel den Rekursiver-Abstieg Parser, das eines der einfachsten Varianten eines Parsers ist und auch ohne Werkzeuge von Hand schreibbar ist\cite{mossenbock:2024}.
Demgegenüber stehen eine große Auswahl an Werkzeugen wie Parsergeneratoren, die aus Grammatikdefinitionen den Code zum parsen automatisch generieren.
Diese haben den Vorteil dass sie hart-zu-implementierende Parservarianten, z.B. Bottom-Up Parser, automatisch implementieren.
Beispiele für solche Werkzeuge sind YACC und ANTLR, die aus einer Grammatikdefinition in einer Notation wie zum Beispiel BNF / \acfi{EBNF} den Parsercode generieren.

\section{Top-Down Parser}

Der Top-Down Parser baut den Syntax Baum aus der Wurzel der Grammatikdefinition heraus und bricht dann die Tokens immer mehr im Detail runter\cite{meduna2007elements}.\\
Um dies zu verdeutlichen erstellen wir uns eine einfache Grammatikdefinition und wenden diese auf einen Eingabesatz an.\\
\begin{figure}[H]
  \begin{align*}
    A &\quad\rightarrow\quad a\ R\ b \\
    R &\quad\rightarrow\quad y\ |\ z
  \end{align*}
  \caption{Beispiel Grammatikdefinition}
  \label{fig:theory:parser:topdown:grammar}
\end{figure}

In der Definition ist $A$ die Wurzel der Grammatik, die mit dem Token $a$ started und mit $b$ endet.
Dazwischen ist eine weitere Grammatikregel $R$ referenziert.
Diese besteht aus einer Entweder-Oder Regel, hier entweder das Token $y$ oder $z$.\\
$A$, $R$ sind hier \textit{Nichtterminalsymbole}, die nur in Zwischenschritten während des Parsens vorkommen.
$a$, $b$, $y$, $z$ sind \textit{Terminalsymbole}, die nicht weiter heruntergebrochen werden.\\
Beispielshaft übergeben wir einem Parser in der Funktionsweise des rekursiven Abstiegs mit dieser Grammatik den Eingabesatz $a y b$.\\
Der Parser liest das erste Token $a$, der mit dem ersten Token der Regel $A$ übereinstimmt.
\begin{figure}[H]
  \centering
  \begin{forest}
    [$A$
      [$a$]
      [$R$ ?]
      [$b$ ?]
    ]
  \end{forest}
  \caption{Das erste Token $a$ wird gelesen und $A$ geöffnet}
\end{figure}
Die Regel $A$ wird geöffnet und das nächste Token $y$ gelesen.
da $A$ jetzt nur $R$ als mögliche Regel hat, wird $R$ geöffnet und mit $y$ aus dem Eingabesatz überprüft, der in $y | z$ enthalten ist.
Zuletzt wird das Token $b$ gelesen und damit die Wurzel Regel $A$ abgeschlossen.
Da auch der Eingabesatz abgearbeitet ist, ist dieser gültig geparst worden und der Syntaxbaum vollständig\cite[S.219]{aho:2006}.\\

\begin{figure}[H]
  \centering
  \begin{forest}
    [$A$
      [$a$]
      [$R$
        [?]
      ]
      [$b$ ?]
    ]
  \end{forest}
  \qquad
  \qquad
  \begin{forest}
    [$A$
      [$a$]
      [$R$
        [$y$]
      ]
      [$b$ ?]
    ]
  \end{forest}
  \qquad
  \qquad
  \begin{forest}
    [$A$
      [$a$]
      [$R$
        [$y$]
      ]
      [$b$]
    ]
  \end{forest}
  \caption{Die weiteren Tokens werden bis zur Vervollständigung des Syntaxbaums gelesen}
\end{figure}

Diese einfache Art des Parsens hat aber einige Restriktionen, die durch die Grammatik eingehalten werden müssen.\\
So darf die Grammatik keine Linksrekursion enthalten, da der Parser sonst in eine Endlosschleife geraten kann.
Will man, wie in folgender Grammatikregel, $A$ so erweitern, dass der Eingabesatz beliebig oft mit $R$ und $b$ forgeführt werden kann:

\begin{figure}[H]
  \begin{align*}
    A &\quad\rightarrow\quad (A\ |\ a)\ R\ b \\
    R &\quad\rightarrow\quad y\ |\ z
  \end{align*}
  \caption{Linksrekursive Grammatikdefinition}
\end{figure}

Kann der Parser im rekursiven Abstieg die Regel $A$ nicht mehr mit dem Eingabetoken lösen; Er muss dazu in $A$ absteigen, aber damit rekursiv immer wieder $A$ lösen.\\
Oftmals lassen sich solche Linksrekursionen durch Umformung der linksrekursiven Regel vermeiden.
In diesem Fall kann $A$ in zwei Regeln aufgeteilt werden, wie in Abbildung \ref{fig:theory:parser:topdown:recursive} dargestellt.
$END$ entspricht hier dem Ende des Eingabesatzes, also ob der Parser noch ein Token lesen konnte oder bereits alle Token eingelesen hat.\\
Dies ist jedoch nicht immer möglich.
Eine mögliche Lösung ist die Grammatikdefinition zu vereinfachen;
Allerdings kann man auch eine andere Parservariante, zum Beispiel die eines Bottom-Up Parsers, benutzen, die mit Linksrekursion umgehen kann (aber ihre eigenen Vor- und Nachteile besitzt).

\begin{figure}[H]
  \begin{align*}
    A &\quad\rightarrow\quad a\ B \\
    B &\quad\rightarrow\quad R\ b\ (END\ |\ B) \\
    R &\quad\rightarrow\quad y\ |\ z
  \end{align*}
  \caption{Aufgelöste linksrekursive Grammatikdefinition}
  \label{fig:theory:parser:topdown:recursive}
\end{figure}

\section{Bottom-Up Parser}

Der Bottom-Up Parser baut den Syntax Baum von den Blättern der Grammatikdefinition zu der Wurzel auf\cite{meduna2007elements}.
Mit der Grammatikdefinition aus Abbildung \ref{fig:theory:parser:topdown:grammar} und dem Eingabesatz $a y b$ liest der Parser das erste Token $a$ ein, das noch nicht zu einer Regel zugeordnet werden kann.
Der Parser speichert dieses Token zwischen und liest das nächste Token $y$ ein.
Das kann direkt der Regel $R$ zugeordnet werden.
Mit dem letzten Token $b$ kann der Parser dann aus dem existierenden Teil des Baumes und diesem Token in die Regel $A$ zusammenfassen.

\begin{figure}[H]
  \centering
  \begin{forest}
    [?
      [$a$]
    ]
  \end{forest}
  \qquad
  \qquad
  \begin{forest}
    [?
      [$a$
        [$a$]
      ]
      [$R$
        [$y$]
      ]
    ]
  \end{forest}
  \qquad
  \qquad
  \begin{forest}
    [$A$
      [$a$
        [$a$]
      ]
      [$R$
        [$y$]
      ]
      [$b$
        [$b$]
      ]
    ]
  \end{forest}
  \caption{Im Bottom-Up Verfahren wird der Syntaxbaum nach und nach von den Blättern aufgebaut}
\end{figure}

Der Bottom-Up Parser \textit{reduziert} hier mithilfe der Tokens die Grammatikregeln zu einen Syntaxbaum, im Gegensatz zu dem Top-Down Parser, der mithilfe der Tokens den Syntaxbaum aus den Grammatikregeln \textit{ableitet}.

\subsection{Shift-Reduce Parser}

Eine Variante des Bottom-Up Parsers ist der ``Shift-Reduce'' Parser, der die Tokens in einem Zwischenspeicher speichert und die Grammatikregeln aus den gelesenen Tokens und dem Zwischenspeicher reduziert.
Im Gegensatz zu der Top-Down Parser Familie sind dessen Grammatikdefinitionen weniger beschränkt, er behandelt zum Beispiel linksrekursive Grammatiken ohne Probleme.
Allerdings ist es schwerer die Semantikanalyse in den Parser zu integrieren, und die Implementation des Parsers selber ist wesentlich komplexer.
So benötigt der Shift-Reduce Parser eine Parsertabelle.
Ausserdem muss der Parser vorher geparste Token und daraus reduzierte Regeln zwischenspeichern, um diese im späteren Verlauf zu weiteren Teilen des Syntaxbaumes kombinieren zu können.
Deswegen werden reale Bottom-Up Parser meistens von Parsergeneratoren aus Grammatikdefinitionen wie der \acfi{BNF} generiert\cite{mossenbock:2024}.\\

Der Shift-Reduce Parser wird, wie viele Parser von kontextfreien Grammatiken, als Kellerautomat implementiert. 
Er speichert die Tokens, die er einliest, in einem Stack (``Keller''), und reduziert sie während des parsens aus der Grammatikdefinition zu dem Syntaxbaum.
Ein Shift-Reduce Parser arbeitet grundlegend mit vier Operationen während des Einlesens der Tokens\cite{mossenbock:2024}:

\begin{itemize}
  \item \textbf{shift}: Das gelesene Token wird vom Eingabesatz in den Keller verschoben.
  \item \textbf{reduce}: Tokens im Keller werden zu Nichtterminalsymbole reduziert
  \item \textbf{accept}: Der Eingabesatz ist vollständig gelesen und der Syntaxbaum vollständig aufgebaut.
  \item \textbf{error}: Der Parser kann den Eingabesatz nicht weiter auf die Grammatikdefinition anwenden, der Eingabesatz ist fehlerhaft
\end{itemize}


Als Beispiel wird wieder der Eingabesatz $a\ y\ b\ \#$ (wobei $\#$ für das Ende des Eingabesatzes steht) mit der Grammatikdefinition aus Abbildung \ref{fig:theory:parser:topdown:grammar} geparst.

\begin{table}[H]
  \centering
  \begin{tabular}{r|l|l}
    Keller        & Eingabe                        & Aktion                    \\
    \qquad\qquad\qquad\qquad        & \qquad\qquad\qquad\qquad\qquad                        & \qquad                    \\
                  & \textit{\textbf{a}}\ \ $y$\ \ $b$\ \ $\#$ & $shift$                   \\
    $a$           & \textit{\textbf{y}}\ \ $b$\ \ $\#$      & $shift$                   \\
    $a$\ \ $y$      & $b$ \ \ $\#$                    & $reduce$ $y$ \rightarrow\ $R$       \\
    $a$\ \ $R$      & \textit{\textbf{b}}\ \ $\#$           & $shift$                   \\
    $a$\ \ $R$\ \ $b$ & $\#$                          & $reduce$ $a\ R\ b$ \rightarrow\ $A$ \\
    $A$           & \textit{\textbf{\#}}                          & $accept$                 
  \end{tabular}
  \caption{Ablauf des Shift-Reduce Parsers}
  \label{fig:theory:parser:shiftreduce}
\end{table}

In der Tabelle \ref{fig:theory:parser:shiftreduce} sind die einzelnen Schritte des Parsers aufgeführt.
Zuerst wird das Token $a$ gelesen und in den Keller verschoben ($shift$).
Da aus diesem Token kein Nichtterminalsymbol reduziert werden kann, wird das nächste Token $y$ gelesen und ebenfalls gekellert.
$y$ hingegen kann in der Grammatikdefinition als $R$ reduziert werden, was in Schritt 3 mit $reduce$ durchgeführt wird.
$a$ bleibt weiterhin unverändert im Keller, nur die Tokens zur Regel $R$ werden mit dem Nichtterminalsymbol $R$ im Keller ersetzt.
Mit dem Lesen von Token $b$ in Schritt 4 können wir alle Tokens im Keller mit der Regel für $A$ in Schritt 5 reduzieren.
Der Parser ist dann am Ende des Eingabesatzes angekommen und konnte die Tokens vollständig zu der Wurzel in der Grammatikdefinition reduzieren.\\

Eine reale Implementation des Kellerautomaten wird mithilfe von States, in die sich der Parser befinden kann, gelöst.
Die Übergänge zwischen den States werden in einer Parsertabelle festgehalten:

\begin{table}[h]
\centering
\begin{tabular}{|c|c|c|c|c|c|c|c|}
State & $a$ & $b$ & $y$ & $z$ & $\#$ & $A$ & $R$ \\\hline
0     & s1        &           &           &          &          & s5 &   \\
1     &           &           & s2        & s2       &          &   & s3 \\
2     &           & rR        &           &          &          &   &   \\
3     &           & s4        &           &          &          &   &   \\
4     &           &           &           &          & rA       &   &   \\
5     &           &           &           &          & acc      &   &   \\
\end{tabular}
\caption{Shift-Reduce Parser Tabelle}
\end{table}

Die States werden im Keller gespeichert und ähnlich wie in der Tabelle \ref{fig:theory:parser:shiftreduce} bei Reduzierungen
im Keller entfernt.
Die Kombination aus State und Token gibt an, welche Aktion der Parser durchführen soll.\\
Mit $sx$ wird das jetzige Zeichen geshifted und der Zustand $x$ an den Keller angehängt.\\
$rX$ reduziert das Nichtterminalsymbol $X$ und entfernt die States der Token von $X$, die am Ende des Kellers stehen.
$X$ wird dann benutzt, um den nächsten Stateübergang herbeizuführen.\\
$acc$ beendet den Parser, wenn das Ende des Eingabesatzes ($\#$) erreicht wird, erfolgreich.\\
Leere Tabellenfelder sind Fehlerzustände, bei denen der Eingabesatz ungültig ist.\\

Angewendet auf unser Beispiel $a\ y\ b\ \#$ ergibt sich folgender Ablauf:

\begin{table}[H]
  \centering
  \begin{tabular}{r|l|l|l}
    Keller                   &             &  Eingabe                        & Aktion                    \\
    \qquad\qquad\qquad\qquad &             &  \qquad\qquad\qquad\qquad\qquad                        & \qquad                    \\
               \textbf{0}    &             &  \textit{\textbf{a}}\ \ $y$\ \ $b$\ \ $\#$ & $s1$                   \\
    0\         \textbf{1}    &             &  \textit{\textbf{y}}\ \ $b$\ \ $\#$      & $s2$                   \\
    0\ 1\      \textbf{2}    &             &  \textit{\textbf{b}} \ \ $\#$                    & $rR$       \\
    0\ 1\                    & \textbf{R}  &  \textit{b} \ \ $\#$                    & $s3$       \\
    0\ 1\      \textbf{3}    &             &  \textit{\textbf{b}}\ \ $\#$           & $s4$                   \\
    0\ 1\ 3\   \textbf{4}    &             &  \textit{\textbf{\#}}           & $rA$                   \\
                       0     & \textbf{A}  &  \textit{\#}           & $s5$                   \\
    0\         \textbf{5}    &             &  \textit{\textbf{\#}}                          & $acc$
  \end{tabular}
  \caption{Stateübergänge des Shift-Reduce Parsers}
  \label{fig:theory:parser:shiftreducestate}
\end{table}

In der Parsertabelle ist gut zu erkennen, dass jeder State für ein Token maximal einen Übergang hat.
Das ist nur bei simplen Grammatiken der Fall.
Man bezeichnet diese Grammatiken als $LR(0)$ Grammatik.
Das $LR$ rührt vom Bottom-Up Parser, der von \textit{L}inks nach \textit{R}echts parst und deswegen auch LR-Parser genannt wird. 
Mit der $0$ wird angegeben, dass der Parser kein ``Vorgriffssymbol'' benötigt, also nicht mehr als ein Token aus dem Eingabesatz im Voraus lesen muss, um den Syntaxbaum abzuleiten.\\
Komplexere Grammatiken werden $LR(n)$ genannt, bei der $n$ die Anzahl der Vorgriffssymbole angibt, die ein Parser zum erfolgreichen Parsen der Grammatik benötigt.
Diese werden nochmals in eine Subkatetegorie $LALR(n)$ eingeteilt, bei denen unter kleinen Einschränkungen in der Grammatikdefinition die Menge an States in der Parsertabelle und somit der Kellerautomat deutlich vereinfacht werden kann.\\
Weit verbreitete Parsergeneratoren wie ``YACC'' und ``GNU Bison'' generieren aus diesen $LALR(1)$ Grammatiken den Bottom-Up Parser.
Sie nutzen Vorteile wie die vereinfachte Fehlerbehandlungen durch die Parsertabelle und dem Fakt das $LALR(1)$ Grammatiken mächtiger sind als $LL(1)$ Grammatiken, die für Top-Down Parser verwendet werden.\\
