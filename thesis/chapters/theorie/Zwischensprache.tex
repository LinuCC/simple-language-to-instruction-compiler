\chapter{Die Zwischensprache im Compiler mit Zweipass Verfahren}
\label{chap:theory:inBetweenLayer}

Die Zwischensprache trennt das Compilerfrontend, das den Quelltext einliest und parst, von dem Compilerbackend, welches den Maschinencode generiert.
Eine gute Zwischensprache abstrahiert die Details vom Quelltext und die des Maschinencodes der Zielarchitektur, sodass die Frontends und Backends nicht voneinander abhängig sind.
Damit kann man ein beliebiges Frontend mit einem beliebigen Backend kombinieren, und muss die jeweilige Komponente nur einmal schreiben.\\
Das Aussehen der Zwischensprache kann viele Varianten annehmen.
So ist zum Beispiel ein Syntaxbaum, der beim Parsen des Quelltextes aufgebaut wird, bereits eine mögliche Zwischensprache.
Sie kann aber auch eher Maschinencode-nah sein und bereits auf einzelne Operationen und Register verweisen.
Für Optimierungen des Kompilates kann es sogar sinnvoll sein, mehrere Zwischensprachen zu verwenden, die aufeinander aufbauen und immer mehr Details des Maschinencodes abbilden.\\
Es gilt hier einen passenden Kompromiss zu finden, der die Optimierungsmöglichkeiten nicht einschränkt, aber auch nicht zu komplex wird.\\

\subsection{Drei-Adressen Code}

Eine Repräsentationsmöglichkeit von Operationen in einer Zwischensprache ist der ``Drei-Adressen Code'' (en. ``Three-Address Code'').
Hier werden namensgebend drei Adressen in einer Operation mit genau einem Operator verwendet\cite{aho:2006}.
Komplexere Operationen werden dadurch in mehrere dieser Operationen aufgeteilt, die dann leichter zu Maschinencode übersetzt werden können.\\
Eine mathematische Operation wie \texttt{x = a + (y * z)} wird in zwei Drei-Adressen Code Operationen umgewandelt:
\begin{figure}[H]
  \begin{align*}
    \texttt{x} &\texttt{\ = y * z}\\
    \texttt{a} &\texttt{\ = b + z}
  \end{align*}
\end{figure}

Da wir hier noch im Bereich der Zwischensprache sind, können die Adressen abstrakter dargestellt werden.
Sie können eine Konstante oder eine Referenz auf eine Variable in der Symboltabelle sein.
Hier ist es auch möglich andere Typen, wie zwischenzeitlich generierte Optimierungen, einzufügen\cite{aho:2006}.\\

Drei-Adressen codes sind nicht nur auf mathematische Operationen beschränkt;
Sie können viele andere Operationen wie Kontrollstrukturen oder Funktionsaufrufe abbilden.
Ein konditionaler Sprung kann mit einem Operator und drei Adressen abgebildet werden.
Operator und zwei Adressen werden für die Bedingung genutzt, die dritte Adresse zeigt auf den Ort, wohin gesprungen wird:

\begin{figure}[H]
  \begin{align*}
    \texttt{\text{if } x < y \text{ goto } T}
  \end{align*}
\end{figure}

Mit dem Drei-Adressen code als Grundeinheit kann die Zwischensprache damit bereits viele Funktionalitäten eines Programmes abbilden ohne große und komplexe Strukturen zu benötigen.\\
Nehmen wir den Ausdruck eines nicht-trivialen Quelltextes

\begin{figure}[H]
  \begin{align*}
    &\texttt{\text{int } x = 0;}\\
    &\texttt{\text{int } y = 1;}\\
    &\texttt{\text{while } x < 10 \text{ \{}}\\
    &\texttt{\quad\text{if } x\ /\ 2\ ==\ 0 \text{ \{}}\\
    &\texttt{\quad\quad x = x + y;}\\
    &\texttt{\quad\text{\}}}\\
    &\texttt{\text{\}}}
  \end{align*}
\end{figure}

dann kann dieser in Drei-Adressen Code umgewandelt werden:



\begin{figure}[H]
  \begin{align*}
    \text{1: }\quad &\texttt{x = 0}\\
    \text{2: }\quad &\texttt{y = 1}\\
    \text{3: }\quad &\texttt{t1 = x < 10}\\
    \text{4: }\quad &\texttt{\text{if } t1 \text{ goto } 6}\\
    \text{5: }\quad &\texttt{\text{goto } 12}\\
    \text{6: }\quad &\texttt{t2 = x\ /\ 2}\\
    \text{7: }\quad &\texttt{t3 = t2 == 0}\\
    \text{8: }\quad &\texttt{\text{if } \text{not}\ t3 \text{ goto } 11}\\
    \text{9: }\quad &\texttt{t4 = x + y}\\
    \text{10: }\quad &\texttt{x = t4}\\
    \text{11: }\quad &\texttt{\text{goto } 3}\\
    \text{12: }\quad &\texttt{\text{end}}
  \end{align*}
\end{figure}

Hier fällt auf, dass diese Repräsentation nicht allzu weit entfernt von Assemblercode tatsächlicher Prozessorarchitekturen ist, aber noch unabhängig von Hardware-spezifischen Konzepten wie Register.\\
Dieser Code hat bereits Optimierungspotential, so können Zeile 7 und 8 kombiniert werden indem der Vergleichsoperator von \texttt{t2 == 0} direkt negiert wird statt das Ergebnis \texttt{t3} zwischenzuspeichern und dann zu negieren.\\
Das Sprungziel von Zeile 8 ist eine Zeile die auch einen nicht-konditionalen Sprung initiiert, daher ist das umschreiben von
\begin{figure}[H]
  \centering
  \texttt{\text{if} \text{not}\ t3 \text{goto} 11}
\qquad
zu\qquad
  \texttt{\text{if} \text{not}\ t3 \text{goto} 3}
\end{figure}
in Zeile 8 eine weitere Möglichkeit.
