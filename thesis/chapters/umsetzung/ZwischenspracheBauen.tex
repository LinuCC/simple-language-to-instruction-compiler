\chapter{Umsetzung der Zwischensprache}
\label{chap:build:inBetweenState}

Die Zwischensprache im Compiler besteht aus der Symboltabelle, die definierten Variablen und Funktionen speichert, und dem Drei-Adressen Code, der die Operationen auflistet.\\
Da die Datentypen dieser Konzepte die Schnittstelle zwischen Frontend und Backend darstellen, werden sie in dem globalen Namespace im Compiler definiert.
Die Compiler-Frontends generieren bei der Analyse des Quelltextes die Daten, die die Symboltabelle und den Drei-Adressen Code repräsentieren, der dann an das Compiler-Backend weitergegeben und dort zu Maschinencode weiterverarbeitet wird.\\


\section{Die Symboltabelle}

Die Symboltabelle speichert alle Attribute, die zu einer Deklaration eines Symbols gehören.
Als Datentyp verwendet der \ac{SLICC} Compiler eine Liste \texttt{std::list} auf ein \texttt{struct}, welches alle potenziellen Attribute eines Symbols speichert.\\
Die Attribute sind:

\begin{itemize}
  \item \textbf{name}: Der Name des Symbols, z.B. der Variablenname, Funktionsname oder Programmname
  \item \textbf{type}: Der Typ des Symbols aus einem enum, z.B. eine Variable vom Typ \texttt{SymbolType::INT}, eine Funktion vom Typ \texttt{SymbolType::FUNC} oder ein Funktionsargument vom Typ \texttt{SymbolType::FUNC\_ARG}
  \item \textbf{line}: Die Zeile, in der das Symbol definiert wurde
  \item \textbf{col}: Das Zeichen in der Zeile, in der das Symbol definiert wurde
  \item \textbf{element\_count}: Die Größe eines Arrays, wenn das Symbol ein Array ist
  \item \textbf{parent}: Der Name des Elternsymbols, z.B. der Funktionsname, in der das Symbol definiert wurde
  \item \textbf{func\_arg\_position}: Die Position des Arguments in der Funktionsdeklaration, wenn das Symbol ein Funktionsargument ist
  \item \textbf{is\_ref}: Ein boolscher Wert, der angibt, ob das Funktionsargument ein Call-by-Reference Argument ist
\end{itemize}

Damit Einträge aus dem Drei-Adressen Code eindeutig Einträge in der Symboltabelle referenzieren können, wird die Kombination aus Name (\texttt{name}) und Elternsymbol (\texttt{parent}) als eindeutiger Schlüssel verwendet.
Theoretisch kann die Symboltabelle durch das Elternsymbol mehrere Ebenen von Symbolabhängigkeiten speichern (\texttt{parent1 < parent2 < mySym}), aber in der Praxis sind die Frontends durch den eindeutigen Schlüssel auf eine Ebene beschränkt.
Theoretisch könnten sie die Hierarchie manuell abbilden (z.B. durch \texttt{mySym.parent = "parent1:parent2"}), was aber im Backend auch unterstützt werden müsste und diese miteinander koppeln würde.\\

\section{Der Drei-Adressen Code}

Der Drei-Adressen Code ist eine Liste \texttt{std::list} von mit \texttt{struct} repräsentierten Operationen, die in der Listen-Reihenfolge ausgeführt werden.

\begin{lstlisting}[language=C++, caption={Drei-Adressen Code Eintrag}]
typedef struct _TacEntry {
  TacOperation op;
  TacArg* arg1;
  TacArg* arg2;
  char *res_ref;
} TacEntry;
\end{lstlisting}

Die Struct besteht aus dem Operationstyp \texttt{op}, dargestellt durch den enum \texttt{TacOperation}, den beiden Argumenten \texttt{arg1} und \texttt{arg2} sowie einer Referenz für das Ergebnis \texttt{result\_ref}.
Je nach Operationstyp werden nur bestimmte Teile der Argumente verwendet, um die Operation auszuführen:

\begin{itemize}
  \item \textbf{ADD, SUB, MUL, DIV, MOD} arithmetische Operationen: \\ \texttt{result\_ref := arg1 op arg2}
  \item \textbf{LT, GT, LE, GE, EQ, NE} Vergleichsoperationen: \\ \texttt{result\_ref := arg1 op arg2}\\ 0 wird für \texttt{true} und 1 für \texttt{false} zurückgegeben
  \item \textbf{ASSIGN} Zuweisung eines Werts zu einem Symbol: \\ \texttt{result\_ref := arg1}
  \item \textbf{GOTO} Sprung zu einem Label: \\ \texttt{goto result\_ref}
  \item \textbf{LABEL} Sprungziel: \\ \texttt{label result\_ref}
  \item \textbf{COND\_GOTO} Bedingter Sprung zu einem Label: \\ \texttt{if arg1 == arg2 goto result\_ref}
  \item \textbf{COND\_NOT\_GOTO} Bedingter Sprung zu einem Label:\\ \texttt{if arg1 != arg2 goto result\_ref}\\ 
    Diese Negation des \texttt{COND\_GOTO} generiert für manche Kontrollstrukturen verständlicheren Drei-Adressen Code als direkt \texttt{COND\_GOTO} zu nutzen.
    Sonst müsste die die vorherige Vergleichsoperation, die bei diesen Kontrollstrukturen generiert wird, immer negiert werden.
  \item \textbf{CALL} Ruft eine Funktion auf, indem es zu diesem Label springt: \\ \texttt{result\_ref := arg1.var\_ref}
  \item \textbf{RET} Beendet den ausgeführten Funktionsaufruf, gibt einen Wert zum \textbf{CALL} zurück wenn gesetzt: \\ \texttt{return valueof result\_ref}
  \item \textbf{REF} Verlinkt ein neues Symbol zu einem existierenden Symbol, sodass beide auf die gleiche Adresse zeigen: \\ \texttt{result\_ref := same\_value\_as arg1.var\_ref}
\end{itemize}

Die beiden Argumente \texttt{arg1} und \texttt{arg2} werden durch ein \texttt{union} abgebildet, die entweder eine Referenz auf ein Symbol \texttt{sym\_ref} oder einen Integer-Wert \texttt{int\_val} speichert.

\begin{lstlisting}[language=C++, caption={Drei-Adressen Code Argument}]
typedef union _TacArg {
  char *var_ref;
  int int_val;
} TacArg;
\end{lstlisting}

\subsection{Generierung des Drei-Adressen Codes}

Um die höheren Konzepte des Slang Quelltextes in den Drei-Adressen Code zu übersetzen, müssen die Frontends aus den Deklarationen und Statements des Quelltextes die passenden Operationen generieren.

\subsubsection{Variablendeklaration}

\begin{lstlisting}[caption={Beispielzuweisungen}]
int x = 5;
int y;
int[3] z;
\end{lstlisting}

Alle Variablen, die deklariert werden, benötigen nur eine \ac{TAC} Operation, die die Zuweisung eines Wertes zu einem Symbol darstellt.
Da die Werte der \texttt{int} Variablen auf den angegebenen Wert initialisiert oder bei Deklaration auf 0 gesetzt werden, wird für jede Deklaration eine \texttt{ASSIGN} Operation generiert.\\
Bei der Deklaration eines Arrays wird jedem Element des Arrays der Wert 0 zugewiesen.\\

\begin{lstlisting}[caption={Drei-Adressen Code für Variablendeklaration}]
ASSIGN, arg1: {int_val: 5}, arg2: NULL, result_ref: "x"

ASSIGN, arg1: {int_val: 0}, arg2: NULL, result_ref: "y"

ASSIGN, arg1: {int_val: 0}, arg2: NULL, result_ref: "z[0]"
ASSIGN, arg1: {int_val: 0}, arg2: NULL, result_ref: "z[1]"
ASSIGN, arg1: {int_val: 0}, arg2: NULL, result_ref: "z[2]"
\end{lstlisting}

\subsubsection{Zuweisungen}

\begin{lstlisting}[caption={Beispielzuweisungen}]
x = 5;
y = x + 2 * 3;
z[1] = 3;
\end{lstlisting}

Zuweisungen funktionieren ähnlich wie Variablendeklarationen, aber anstatt den Wert \texttt{0} zu initialisieren, wird der Wert des Ausdrucks auf der rechten Seite der Zuweisung berechnet und zugewiesen.\\
Die Variable \texttt{y} hat eine nicht-triviale Berechnung, die in mehrere Operationen aufgeteilt wird.\\
Slang definiert keine Priorität der Operationsreihenfolge, daher wird die Reihenfolge der Operationen durch die Reihenfolge der Argumente bestimmt.\\

\begin{lstlisting}[caption={Drei-Adressen Code für Zuweisungen}]
ASSIGN, arg1: {int_val: 5}, arg2: NULL, result_ref: "x"

ADD, arg1: {var_ref: "x"}, arg2: {int_val: 2}, result_ref: ":t1"
MUL, arg1: {var_ref: ":t1"}, arg2: {int_val: 3}, result_ref: "y"

ASSIGN, arg1: {int_val: 3}, arg2: NULL, result_ref: "z[1]"
\end{lstlisting}

Sobald Operationen in mehrere Operationen aufgeteilt werden, müssen temporäre Symbole Zwischenzustände speichern.
Hier wird ein temporäres Symbol \texttt{:t1} verwendet, dass in die Symboltabelle eingetragen wird, um für das nächste \texttt{MUL} wieder benutzt werden zu können.

\subsubsection{if-Bedingungen}

\begin{lstlisting}[caption={Slang Vorlage if-Bedingung}]
if (<comp_expr>) {
  <body>
} else {
  <else_body>
}
\end{lstlisting}

Die Blöcke der \texttt{if}-Bedingungen werden mithilfe von \texttt{GOTO}s und \texttt{LABEL}s auseinandergehalten.

\begin{lstlisting}[caption={Drei-Adressen Code Vorlage für if-Bedingung}]
<comp_expr>, result_ref: ":t1"
COND_NOT_GOTO, arg1: {var_ref: ":t1"}, arg2: {int_val: 0}, result_ref: "ELSE_LABEL"

<body>

GOTO, arg1: NULL, arg2: NULL, result_ref: "ELSE_END_LABEL"
LABEL, arg1: NULL, arg2: NULL, result_ref: "ELSE_LABEL"

<else_body>

LABEL, arg1: NULL, arg2: NULL, result_ref: "ELSE_END_LABEL"

\end{lstlisting}

Für \texttt{if}-Bedingungen wird ein bedingter Sprung zu \texttt{ELSE\_LABEL} generiert, der ausgeführt wird, wenn der Vergleich nicht \texttt{0} zurückliefert.
Ist dies nicht der Fall, wird der \texttt{<body>} ausgeführt und am Ende des \texttt{<body>} mit einem \texttt{GOTO ELSE\_END\_LABEL}  der \texttt{<else\_body>} übersprungen.\\
Wird der bedingte Sprung ausgeführt, wird der \texttt{<else\_body>} ausgeführt.
Am Ende des \texttt{<else\_body>} wird ein \texttt{ELSE\_END\_LABEL} eingefügt, damit bei Ausführung des \texttt{<body>} der \texttt{<else\_body>} übersprungen werden kann.\\

\subsubsubsection{Beispiel:}

\begin{lstlisting}[caption={Beispiel if-Bedingung}]
if (x < 5) {
  y = 5;
}
\end{lstlisting}

\begin{lstlisting}[caption={Drei-Adressen Code für die if-Bedingung}]
LT, arg1: {var_ref: "x"}, arg2: {int_val: 5}, result_ref: ":t1"
COND_NOT_GOTO, arg1: {var_ref: ":t1"}, arg2: {int_val: 0}, result_ref: "L1"

ASSIGN, arg1: {int_val: 5}, arg2: NULL, result_ref: "y"

LABEL, arg1: NULL, arg2: NULL, result_ref: "L1"
\end{lstlisting}

\subsubsection{for-Schleifen}

\begin{lstlisting}[caption={Beispielschleife}, label={lst:forloop}]
for (<comp_expr>; <assign_expr>) {
  <body>
}
\end{lstlisting}

ie Vorlage für die For-Schleife mit den Platzhaltern aus \autoref{lst:forloop} hat die folgende Struktur:

\begin{lstlisting}[caption={Drei-Adressen Code für for-Schleifen}]
LABEL, arg1: NULL, arg2: NULL, result_ref: "START_LABEL"
<comp_expr>, result_ref: ":t1"
COND_NOT_GOTO, arg1: {var_ref: ":t1"}, arg2: {int_val: 0}, result_ref: "END_LABEL"

<body>
<assign_expr>

GOTO, arg1: NULL, arg2: NULL, result_ref: "START_LABEL"

LABEL, arg1: NULL, arg2: NULL, result_ref: "END_LABEL"
\end{lstlisting}

Die Bedingung der \texttt{for}-Schleife wird in einem bedingten Sprung abgebildet, der, erst wenn die Bedingung  \texttt{<comp\_expr>} nicht mehr erfüllt ist, zu dem Label \texttt{END\_LABEL} springt, das die Schleife beendet.
Passiert der bedingte Sprung nicht, wird zuerst der Block der Schleife ausgeführt und danach die Zuweisung \texttt{assign\_expr} aus dem Schleifenkopf.
Dann wird zurückgesprungen zum \texttt{START\_LABEL} und die \texttt{<comp\_expr>} sowie der bedingte Sprung erneut ausgeführt.\\

Zum Beispiel wird aus folgender Schleife

\begin{lstlisting}[caption={Beispielschleife}]
int i;

for (i < 5; i = i + 1) {
  x = x + 1;
}
\end{lstlisting}

der folgende Drei-Adressen Code generiert:

\begin{lstlisting}[caption={Drei-Adressen Code für for-Schleifen}]
ASSIGN, arg1: {int_val: 0}, arg2: NULL, result_ref: "i"

LABEL, arg1: NULL, arg2: NULL, result_ref: "L1"
LT, arg1: {var_ref: "i"}, arg2: {int_val: 5}, result_ref: ":t1"
COND_NOT_GOTO, arg1: {var_ref: ":t1"}, arg2: {int_val: 0}, result_ref: "L2"

ADD, arg1: {var_ref: "x"}, arg2: {int_val: 1}, result_ref: ":t2"
ASSIGN, arg1: {var_ref: ":t2"}, arg2: NULL, result_ref: "x"

ADD, arg1: {var_ref: "i"}, arg2: {int_val: 1}, result_ref: "i"
GOTO, arg1: NULL, arg2: NULL, result_ref: "L1"

LABEL, arg1: NULL, arg2: NULL, result_ref: "L2"
\end{lstlisting}

\subsection{Funktionsdeklaration und Funktionsaufruf}

\begin{lstlisting}[caption={Vorlage Funktionsdeklaration und Funktionsaufruf}]
myFunc(<args>);

func <type> myFunc(<arg_types>) {
  <func_body>

  return <type_val>
}
\end{lstlisting}


Funktionsdeklarationen erwarten, dass vordefinierte Funktionsargumente namens \texttt{:args\_x} (mit \texttt{x} als Position des Arguments) in der Symboltabelle vorhanden sind.
Bei einem Funktionsaufruf werden zuerst \acp{TAC} generiert die diese \texttt{:args\_x} auf die Symbole, die als Argumente in den Funktionsaufruf gegeben werden, setzen.
Für Argumente, die nicht per Referenz übergeben werden, wird eine \texttt{ASSIGN} Operation generiert, die den Wert des Funktionsarguments in ein temporäres Symbol speichert, der dann weiter genutzt wird.\\
Für Argumente, die per Referenz übergeben werden, wird eine \texttt{REF} Operation generiert, die ein neues temporäres Symbol generiert, welches auf die gleiche Adresse wie das verlinkte Symbol zeigt.\\
Dann wird mit einem \texttt{CALL} zum \texttt{<func\_body>} gesprungen und dieser weiter ausgeführt.\\
Am Ende des Funktionsaufrufs wird ein \texttt{RET} generiert, das den Wert des \texttt{<type\_val>} (wenn gegeben) zurückgibt.\\
Es kann auch vorher ein \texttt{RET} durch einen frühen \texttt{return} Befehl ausgeführt werden, dann wird die Ausführung des \texttt{<func\_body>} abgebrochen und der Wert dieses \texttt{return} Befehls zurückgegeben.\\

Nehmen wir die folgende Funktionsdeklaration und den Funktionsaufruf als Beispiel:

\begin{lstlisting}[caption={Beispiel Funktionsdeklaration und Funktionsaufruf}]
int x = 5;
int y = 3;
int z = 0;

z = myFunc(x, y);

func int myFunc(int xa, ref int yb) {
  xa = 20;
  yb = 10;
  if (xa == 20) {
    return xa;
  }
}
\end{lstlisting}

Es wird zuerst eine Funktionsdefinition generiert, die den Wert des Arguments \texttt{xa} dupliziert und die Adresse des Arguments \texttt{yb} referenziert.

\begin{lstlisting}[caption={Drei-Adressen Code Funktionsdeklaration}]
GOTO, arg1: NULL, arg2: NULL, result_ref: "LABEL_MY_FUNC_END"

LABEL, arg1: NULL, arg2: NULL, result_ref: "LABEL_MY_FUNC"
ASSIGN, arg1: {var_ref: ":args_0"}, arg2: NULL, result_ref: ":t1"
REF, arg1: {var_ref: ":args_1"}, arg2: NULL, result_ref: ":t2"

ASSIGN, arg1: {int_val: 20}, arg2: NULL, result_ref: ":t1"
ASSIGN, arg1: {int_val: 10}, arg2: NULL, result_ref: ":t2"

// if-Bedingung
EQ, arg1: {var_ref: ":t1"}, arg2: {int_val: 20}, result_ref: ":t3"
COND_NOT_GOTO, arg1: {var_ref: ":t3"}, arg2: {int_val: 0}, result_ref: "L1"
RET, arg1: {var_ref: ":t1"}, arg2: NULL, result_ref: NULL
LABEL, arg1: NULL, arg2: NULL, result_ref: "L1"

// Standard Return falls nichts zurückgegeben wird
RET, arg1: {var_ref: 0}, arg2: NULL, result_ref: NULL

LABEL, arg1: NULL, arg2: NULL, result_ref: "LABEL_MY_FUNC_END"
\end{lstlisting}

Nach der Funktionsdefinition werden die Variablen des Hauptteils deklariert und der Funktionsaufruf generiert, der die Werte der Variablen \texttt{x} und \texttt{y} in die Funktionsargumente \texttt{xa} und \texttt{yb} setzt.
Dabei muss hier schon auf die Argumenttypen, die in der Symboltabelle gespeichert sind, geachtet werden.
\texttt{:args\_1} muss die Adresse für \texttt{y} bereits referenzieren, damit spätere Zuweisungen in der Funktion auch die Variable \texttt{y} ändern.\\

\begin{lstlisting}[caption={Drei-Adressen Code Funktionsaufruf}]

ASSIGN, arg1: {int_val: 5}, arg2: NULL, result_ref: "x"
ASSIGN, arg1: {int_val: 3}, arg2: NULL, result_ref: "y"
ASSIGN, arg1: {int_val: 0}, arg2: NULL, result_ref: "z"

ASSIGN, arg1: {var_ref: "x"}, arg2: NULL, result_ref: ":args_0"
REF, arg1: {var_ref: "y"}, arg2: NULL, result_ref: ":args_1"
CALL, arg1: {var_ref: "LABEL_MY_FUNC"}, arg2: NULL, result_ref: "z"

\end{lstlisting}

Mit diesem Vorlagen können wir nun für alle Konzepte des Slang Quelltextes den Drei-Adressen Code generieren, der optimiert oder direkt weiter an das Compiler-Backend gegeben werden kann.
