\part{Umsetzung des Compilers}
\chapter{Planung Des Compilers}

Die Implementierung des Compilers soll folgende Ziele erfuellen:

\begin{itemize}
\item Verstaendlichkeit und Inspizierbarkeit: Der Compiler soll inspizierbar und verstaendlich aufgebaut sein. Zusaetzlich soll die Arbeitsweise des Compilers transparent sein, indem die einzelnen Teile des Compilers gut debugbar oder inspizierbar sind.
\item Flexibles Backend: Der Compiler soll in der Lage sein, auf verschiedene Zielplattformen wie RISC Assembly und RISC Binary zu kompilieren.
\end{itemize}

Der in dieser Arbeit erstellte Compiler \acfi{SLICC} wird hauptsächlich in C++ geschrieben, da die Werkzeuge wie die Parsergeneratoren, die benutzt werden, gut darauf aufbauen.
Damit der Compiler selbst möglichst interoperabel auf verschiedenen Plattformen kompiliert werden kann, wird das robuste und weit verbreitete ``CMake'' zum bauen der Binärdatei benutzt.\\
Da der Compiler ein Zweipass-Compiler ist, benötigen wird die Auftrennung des Compilers in Frontend und Backend durch eine Zwischensprache.
Die Zwischensprache wird in diesem Fall ein - in dem Theorie-Teil bereits erwähnter - Drei-Adressen Code sein, der die Operationen des Programmcodes in einfache Operationen mit drei Adressen aufteilt.
Diese Zwischensprache trennt ein Frontend, dass eine simple, Pascal-ähnliche Programmiersprache namens \acfi{SLang} parst, von einem Backend, dass den Drei-Adressen Code in lesbaren RISC Assembly oder RISC Binary übersetzt und in eine Datei speichert.\\

\begin{figure}[H]
  \begin{tikzpicture}
  \sffamily
    \node[minimum height=1.5cm,minimum width=2cm,text centered,draw](slangfrontend){Slang Frontend};
    \node[right=0.9cm of slangfrontend, minimum height=1.5cm,minimum width=2cm,text centered, text width=2cm, draw] (state){Three-Address\\
    Code};
    \node[title,left=0.75cm of slangfrontend, text width=2.5cm, text centered](slangcode){Slang\\
    Programmcode};
    \node[title,above=0.33cm of slangfrontend](frontendstitle){Frontends};
    \node[above=1cm of slangfrontend, fit={(slangfrontend)($(slangfrontend)+(0,-1.75)$)},minimum width=2cm,draw=black!50] (frontendsbox);
    \node[right=4cm of slangfrontend, minimum height=1.5cm,minimum width=2.5cm,text centered,draw](riscbackend){RISC Backend};
    \node [title,above=0.33cm of riscbackend](backendstitle){Backends};
    \node[above=1cm of riscbackend, fit={(riscbackend)($(riscbackend.south)+(0,-1)$)},minimum width=2cm,draw=black!50] (backendsbox);
    \node [title,right=0.75cm of riscbackend, text width=2cm, text centered](riscbinary){RISC\\
    Maschinencode};
    \node[above=0.25cm of slangfrontend, fit={(frontendsbox)($(backendsbox.southeast)+(0, -1.75)$)},minimum width=2cm,draw=black!50] (compilerbox);
    \node [title,above=1.5cm of state](compilertitle){SLICC Compiler};
    \umltrans[arg=]{slangcode}{slangfrontend}
    \umltrans[arg=]{riscbackend}{riscbinary}
    \umltrans[arg=]{x86backend}{x86binary}
    \umltrans[arg=]{slangfrontend}{state}
    \umltrans[arg=]{state}{riscbackend}
  \end{tikzpicture}
  \caption{Aufbau des SLICC Compilers}
  \label{fig:slicc:compiler}
\end{figure}

Da hier erstmal nur ein Frontend und ein Backend implementiert werden, muss die Entkopplung in der Implementierung des Compilers sichergestellt werden um die Flexibilität des Projektes zu wahren.\\
Um dies durchzusetzen wird das Submodul-Feature von CMake benutzt um das Frontend und Backend zu Bibliotheken zu kapseln, die dann vom Hauptprogramm, dem ``Driver'', eingebunden werden können.
Als Ordnerstruktur wird folgendes gewählt:

\begin{figure}[H]
\begin{forest}
  for tree={
    font=\ttfamily,
    grow'=0,
    child anchor=west,
    parent anchor=south,
    anchor=west,
    calign=first,
    edge path={
      \noexpand\path [draw, \forestoption{edge}]
      (!u.south west) +(7.5pt,0) |- node[fill,inner sep=1.25pt] {} (.child anchor)\forestoption{edge label};
    },
    before typesetting nodes={
      if n=1
        {insert before={[,phantom]}}
        {}
    },
    fit=band,
    before computing xy={l=15pt},
  }
[slicc/
  [src/
    [slang\_parser/\quad \normalfont{\textit{(Frontend)}}
    [parser.yy \quad \normalfont{\textit{(GNU Bison basierter Parser)}}]
    [scan.ll \quad \normalfont{\textit{(Flex basierter Lexer)}}]
      [CMakeLists.txt \quad \normalfont{\textit{(Generiert }} \texttt{slang\_parser.o})]
    ]
    [risc\_gen/\quad \normalfont{\textit{(Backend)}}
      [code\_gen.cc \quad \normalfont{\textit{(Code Generator)}}]
      [optimizer.cc \quad \normalfont{\textit{(Backend-spezifische Optimierungen)}}]
      [CMakeLists.txt \quad \normalfont{\textit{(Generiert }} \texttt{risc\_gen.o})]
    ]
    [other\_frontend/
      [...]
    ]
    [other\_backend/
      [...]
    ]
    [main.cc]
    [optimizer.cc \quad \normalfont{\textit{(Generische Optimierungen auf Zwischensprache)}}]
  ]
[CMakeLists.txt \quad \normalfont\textit{(Generiert }\texttt{slicc} \textit{Binärdatei)}]
]
\end{forest}
\end{figure}

Dadurch können dann in \texttt{main.cc} die Frontend- und Backend-Bibliotheken eingebunden und je nach Eingabeparameter bei der Ausführung ausgewählt werden.
