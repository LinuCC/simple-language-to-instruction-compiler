\chapter{Umsetzung des RISC-V Backends}
\label{chap:build:backend}

Damit das Backend gültigen RISC-V Maschinencode generieren kann, muss es dem Drei-Adressen Code und der Symboltabelle der Zwischensprache folgende Dinge entnehmen:

\begin{itemize}
  \item{Register und Speicheradressen, zugeschnitten auf die RISC-V Prozessorarchitektur}
  \item{Die RISC Befehle, die die Operationen der Drei-Adressen Operationen abbilden}
  \item{Optimierungen, zum Beispiel die Lebensdauer der temporären Variablen, um Register freizugeben}
\end{itemize}

RISC-V hat 31 Allzweckregister, mit denen gearbeitet werden kann.
Solange die Anzahl der Variablen und temporären Variablen in einem Programm nicht die Anzahl der Register übersteigt, können diese einfach auf die Register verteilt werden.\\
Werden mehr Variablen benötigt, kann man zum einen Optimierungen einbauen, die die Lebensdauer der Variablen analysieren und Register für andere Variablen freigeben; 
zum anderen kann man mit Methoden in der Registerzuteilung arbeiten, um die Variablen in Adressen im Stack zu speichern und, wenn wieder benötigt, in ein Register zu laden.\\
Da dies das Compiler-Backend deutlich verkompliziert, werden hier nur die Register benutzt, um die Werte der Variablen der Drei-Adressen Operationen zu speichern.\\

\section{Register und Speicheradressen}

Im ersten Schritt wird also den Variablen in der Symboltabelle Register zugewiesen.\\
Wenn mehr als 31 temporäre Variablen und Variablen gesetzt sind, brechen wir die Kompilation ab und geben eine Fehlermeldung aus.\\
Wir verwenden \texttt{x0} nicht, da es immer den Wert 0 hat und somit nicht als Register für Variablen genutzt werden kann.\\

\begin{table}[H]
	\begin{center}
		\begin{tabular}{| r | c |}
			\hline
			Variablenname & Register \\
			\hline
			x & x1 \\
			foobar & x2 \\
			:t1 & x3 \\
			:t2 & x4 \\
			y & x5 \\
			:t3 & x6 \\
			\hline
		\end{tabular}
	\end{center}
	\caption{Zuweisung von Variablen in einer fiktiven Symboltabelle zu Registern}
\end{table}

\section{Zuweisung von Drei-Adressen Code zu RISC-V Befehlen}

\subsection{Arithmetische Operationen}

Da die Variablen bereits in Register abgelegt sind, können die arithmetischen Operationen der Drei-Adressen Operationen direkt in RISC-V Befehle umgewandelt werden.\\
Der Befehl \texttt{li} (Load Immediate) wird benutzt, um Konstanten in Register zu laden und somit die Register unserer Variablen zu initialisieren.\\
Mit den Befehlen \texttt{add}, \texttt{sub}, \texttt{mul}, \texttt{div} und \texttt{rem} werden die arithmetischen Operationen durchgeführt.\\
Um den Wert einer Variablen mit einer anderen zu überschreiben, wird der Befehl \texttt{add} benutzt, um \texttt{<reg> + 0} in das zu überschreibende Register zu speichern.\\

\subsubsection{Beispiel Drei-Adressen Code}

\begin{lstlisting}
ASSIGN, arg1: {int_val: 5}, arg2: NULL, result_ref: "x"
ASSIGN, arg1: {int_val: 0}, arg2: NULL, result_ref: "y"

ADD arg1: { var_ref: "x" }, arg2: { int_val: 2 }, result_ref: "x"
SUB arg1: { var_ref: "x" }, arg2: { int_val: 1 }, result_ref: "x"
MUL arg1: { var_ref: "x" }, arg2: { int_val: 2 }, result_ref: "x"
DIV arg1: { var_ref: "x" }, arg2: { int_val: 3 }, result_ref: "x"
MOD arg1: { var_ref: "x" }, arg2: { int_val: 4 }, result_ref: "x"

ASSIGN, arg1: {int_val: x}, arg2: NULL, result_ref: "y"
\end{lstlisting}

Mit folgenden Registern:

\begin{table}[H]
	\begin{center}
		\begin{tabular}{| r | c |}
			\hline
			Variablenname & Register \\
			\hline
			x & x1 \\
			y & x2 \\
			\hline
		\end{tabular}
	\end{center}
\end{table}

\subsubsection{Resultierende RISC-V Befehle}

\begin{lstlisting}
li x1, 5
li x2, 0

add x1, x1, 2
sub x1, x1, 1
mul x1, x1, 2
div x1, x1, 3
rem x1, x1, 4

add x2, x1, 0
\end{lstlisting}

\subsection{Sprünge}

Bedingte Sprünge können in RISC-V direkt mit dem Befehl \texttt{beq} (Branch if Equal) und \texttt{bne} (Branch if Not Equal) umgesetzt werden.\\
Unbedingte Sprünge können mit dem Befehl \texttt{jal} (Jump and Link) umgesetzt werden, wobei von \texttt{jal} zurückgegebene \textit{Return Address} in \texttt{x0} gespeichert und somit ignoriert wird.

\subsubsection{Beispiel Drei-Adressen Code}

\begin{lstlisting}
ASSIGN, arg1: {int_val: 5}, arg2: NULL, result_ref: "x"
ASSIGN, arg1: {int_val: 10}, arg2: NULL, result_ref: "y"

LABEL, label: "LABEL_S"
COND_GOTO arg1: { var_ref: "x" }, arg2: { var_ref: "y" }, target: "LABEL_T"
COND_NOT_GOTO arg1: { var_ref: "x" }, arg2: { int_val: 5 }, target: "LABEL_T"

LABEL, label: "LABEL_T"
GOTO target: "LABEL_S"
\end{lstlisting}

Mit folgenden Registern:

\begin{table}[H]
  \begin{center}
    \begin{tabular}{| r | c |}
      \hline
      Variablenname & Register \\
      \hline
      x & x1 \\
      y & x2 \\
      \hline
    \end{tabular}
  \end{center}
\end{table}

\subsubsection{Resultierende RISC-V Befehle}

\begin{lstlisting}
li x1, 5
li x2, 10

LABEL_S:
beq x1, x2, LABEL_T
bne x1, 5, LABEL_T

LABEL_T:
jal x0, LABEL_S:
\end{lstlisting}

\subsection{Vergleichsoperationen}

Vergleichsoperationen sind durch die geringe Anzahl der Befehlsoperationen etwas interessanter.
Der Befehl \texttt{slt} (Set Less Than) kann für den \texttt{<} (und mit umgedrehten Argumenten für den \texttt{>}) Operator benutzt werden.
RISC-V implementiert kein Befehl für \texttt{<=} und \texttt{>=}, dies wird gelöst, indem der \texttt{slt} Operator benutzt wird und das Ergebnis mit dem Operator \texttt{xori} negiert wird.

\begin{figure}[H]
  \begin{align*}
    Less\ Or\ Equal\ Than\ &\texttt{=\ !(x < y) = x >= y}\\
    More\ Or\ Equal\ Than\ &\texttt{=\ !(y < x) = x <= y}
  \end{align*}
\end{figure}
Ansonsten werden im Allgemeinen die Befehle \texttt{beq} (Branch if Equal) für \texttt{==} und \texttt{bne} (Branch if Not Equal) für \texttt{!=} empfohlen, um Werte zu vergleichen.
Will man dabei einen Wert setzen, kann man dies nach dem entsprechenden Sprungbefehl machen.\\

\subsubsection{Beispiel Drei-Adressen Code}

\begin{lstlisting}
ASSIGN, arg1: {int_val: 5}, arg2: NULL, result_ref: "x"
ASSIGN, arg1: {int_val: 10}, arg2: NULL, result_ref: "y"

LT arg1: { var_ref: "x" }, arg2: { var_ref: "y" }, result_ref: "t1"
GT arg1: { var_ref: "x" }, arg2: { var_ref: "y" }, result_ref: "t2"
LE arg1: { var_ref: "x" }, arg2: { var_ref: "y" }, result_ref: "t3"
GE arg1: { var_ref: "x" }, arg2: { var_ref: "y" }, result_ref: "t4"
EQ arg1: { var_ref: "x" }, arg2: { var_ref: "y" }, result_ref: "t5"
NE arg1: { var_ref: "x" }, arg2: { var_ref: "y" }, result_ref: "t6"
\end{lstlisting}

Mit folgenden Registern:

\begin{table}[H]
	\begin{center}
		\begin{tabular}{| r | c |}
			\hline
			Variablenname & Register \\
			\hline
			x & x1 \\
			y & x2 \\
			t1 & x3 \\
			t2 & x4 \\
			t3 & x5 \\
			t4 & x6 \\
			t5 & x7 \\
			t6 & x8 \\
			\hline
		\end{tabular}
	\end{center}
\end{table}

\subsubsection{Resultierende RISC-V Befehle}

\begin{lstlisting}
li x1, 5
li x2, 10

slt x3, x1, x2

slt x4, x2, x1

slt x5, x2, x1
xori x5, x5, -1

slt x6, x1, x2
xori x6, x6, -1

beq x1, x2, LABEL_EQ_NOT_TRUE
  li x7, 0
  j LABEL_EQ_END
LABEL_EQ_NOT_TRUE:
  li x7, 1
LABEL_EQ_END:

bne x1, x2, LABEL_NE_NOT_TRUE
  li x8, 0
  j LABEL_NE_END
LABEL_NE_NOT_TRUE:
  li x8, 1
LABEL_NE_END:
\end{lstlisting}

RISC-V bietet außerdem einige Pseudo-Operationen an, die zur Lesbarkeit in der Maschinensprache benutzt werden können.
Statt \texttt{xori x5, x5, -1} kann man die Pseudo-Operation \texttt{not x5, x5} benutzen, die zu \texttt{xori} expandiert wird, aber etwas lesbarer ist.\\


\subsection{Funktionsaufrufe}

Funktionsaufrufe können wieder mit dem Befehl \texttt{jal} umgesetzt werden, allerdings muss hierbei die \textit{Return Address} in einem Register gespeichert werden, um nach dem Funktionsaufruf wieder zurückzuspringen.\\
Das macht allerdings verschachtelte Funktionsaufrufe schwierig, da die \textit{Return Address} bei mehreren Funktionsaufrufen ineinander überschrieben wird.\\
Die Lösung hierfür wäre, die Variablen sauber auf Stack-Adressen zu speichern und die Funktionsdefinition mittels eines Funktionsprologs korrekt aufzusetzen.
Dies ist im Rahmen dieser Arbeit zu aufwändig und wird daher nicht implementiert.\\
Stattdessen wird als temporäre Maßnahme die Kompilierung mit einem Fehler abgebrochen, wenn ein Funktionsaufruf in einem Funktionsaufruf auftritt.\\
Die \textit{Return Address} wird in \texttt{x31} gespeichert, das Register wird dabei als belegt aufgenommen.\\
Der Wert, der von der Funktion zurückgegeben wird, wird in dem Register \texttt{x30} gespeichert.\\

\subsubsection{Beispiel Drei-Adressen Code}

\begin{lstlisting}[caption={Drei-Adressen Code Funktionsaufruf}]

ASSIGN, arg1: {int_val: 5}, arg2: NULL, result_ref: "x"
ASSIGN, arg1: {int_val: 3}, arg2: NULL, result_ref: "y"
ASSIGN, arg1: {int_val: 0}, arg2: NULL, result_ref: "z"

ASSIGN, arg1: {var_ref: "x"}, arg2: NULL, result_ref: ":args_0"
REF, arg1: {var_ref: "y"}, arg2: NULL, result_ref: ":args_1"
CALL, arg1: {var_ref: "LABEL_MY_FUNC"}, arg2: NULL, result_ref: "z"

\end{lstlisting}

Mit folgenden Registern:

\begin{table}[H]
  \begin{center}
    \begin{tabular}{| r | c |}
      \hline
      Variablenname & Register \\
      \hline
      x & x1 \\
      y & x2 \\
      z & x3 \\
      :args\_0 & x4 \\
      :args\_1 & x5 \\
      \textit{Return Value} & x30 \\
      \textit{Return Address} & x31 \\
      \hline
    \end{tabular}
  \end{center}
\end{table}

\subsubsection{Resultierende RISC-V Befehle}

\begin{lstlisting}
li x1, 5
li x2, 3
li x3, 0

add x4, x1, 0
add x5, x2, 0
jal x31, LABEL_MY_FUNC
add x3, x30, 0

\end{lstlisting}

\pagebreak

\subsection{Funktionsdefinitionen}

Um die Rücksprünge von Funktionsaufrufen zu ermöglichen, ohne Funktionsprolog und -epilog zu implementieren, wird das Register \texttt{x31} für die Rücksprungadresse benutzt, wobei verhinder wird, dass der Quellcode Funktionsaufrufe verschachtelt.\\
Um zu der Adresse zurückzuspringen, die vor dem Funktionsaufruf gespeichert wurde, wird der Befehl \texttt{jr} (Jump Return) mit dem Register \texttt{x31} benutzt\cite{waterman:2017}.\\
Diese simple RISC Funktionsdefinition ist strukturell ähnlich zu dem Drei-Adressen Code, aus dem sie generiert wurde.

\subsubsection{Beispiel Drei-Adressen Code}

\begin{lstlisting}[caption={Drei-Adressen Code Funktionsdeklaration}]
GOTO, arg1: NULL, arg2: NULL, result_ref: "LABEL_MY_FUNC_END"

LABEL, arg1: NULL, arg2: NULL, result_ref: "LABEL_MY_FUNC"
ASSIGN, arg1: {var_ref: ":args_0"}, arg2: NULL, result_ref: ":t1"
REF, arg1: {var_ref: ":args_1"}, arg2: NULL, result_ref: ":t2"

ASSIGN, arg1: {int_val: 20}, arg2: NULL, result_ref: ":t1"
ASSIGN, arg1: {int_val: 10}, arg2: NULL, result_ref: ":t2"

// if-Bedingung
EQ, arg1: {var_ref: ":t1"}, arg2: {int_val: 20}, result_ref: ":t3"
COND_NOT_GOTO, arg1: {var_ref: ":t3"}, arg2: {int_val: 0}, result_ref: "L1"
RET, arg1: {var_ref: ":t1"}, arg2: NULL, result_ref: NULL
LABEL, arg1: NULL, arg2: NULL, result_ref: "L1"

// Standard Return falls nichts zurückgegeben wird
RET, arg1: {int_val 0}, arg2: NULL, result_ref: NULL

LABEL, arg1: NULL, arg2: NULL, result_ref: "LABEL_MY_FUNC_END"
\end{lstlisting}

\begin{table}[H]
  \begin{center}
    \begin{tabular}{| r | c |}
      \hline
      Variablenname & Register \\
      \hline
      :args\_0 & x4 \\
      :args\_1 & x5 \\
      :t1 & x6 \\
      :t2 & x7 \\
      :t3 & x8 \\
      \textit{Return Address} & x31 \\
      \hline
    \end{tabular}
  \end{center}
\end{table}

\subsubsection{Resultierende RISC-V Befehle}

\begin{lstlisting}
jal x0, LABEL_MY_FUNC_END

LABEL_MY_FUNC:
add x6, x4, 0

li x6, 20
li x5, 10

beq x6, 20, L1

add x30, x6, 0
jr x31

L1:
li x30, 0
jr x31

LABEL_MY_FUNC_END:
\end{lstlisting}
