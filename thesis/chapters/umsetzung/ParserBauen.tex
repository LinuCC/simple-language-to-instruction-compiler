\chapter{Umsetzung des Parsers}
\label{chap:building:parser}

Der angezielte Parser ist ein Bottom-Up Parser generiert durch einen Parsergenerator.
Die lexikalische Analyse wird von ``Flex'', einer erweiterten Alternative zum klassischen ``Lex'', übernommen.
Flex parst eine Datei mit einer zu diesem Werkzeug spezifischen Syntax und schreibt den Lexer als autogenerierten C oder C++ Code in eine Datei.
Mit der Schnittstelle dieses Lexers kann dann gearbeitet werden, um die Tokens während des Parsens weiterzuverarbeiten.\\
Flex bringt eine spezielle Schnittstelle mit, mit der Syntaxparsergeneratoren wie ``Yacc'' oder kompatible Derivate wie ``GNU Bison'' direkt integriert werden können ohne in handgeschriebenen Programmcode verbindet werden zu müssen\cite{flex:1995}.\\
Bison bietet sich auch direkt an da es noch weiterentwickelt wird und ebenfalls das Yacc-kompatible Interface implementiert.
Der Parser von Bison implementiert interessante Möglichkeiten, wie zum Beispiel die Erweiterung des Parsers von \textit{LR(1)} Grammatiken zu \textit{GLR} (Generalized LR) Grammatiken.
Dadurch kann der Parser auch mit Shift-Reduce Konflikten umgehen, indem der Parser mehrere Parse-Stränge gleichzeitig verfolgt und bei Fehlern diese wieder schliesst, bis nur der eine gültige Strang (bei korrekter Grammatikdefinition) weiterverfolgt wird\cite{bisonmanual}.\\

\section{Der Aufbau der Programmiersprache Slang}

Die Programmiersprache Slang ist eine Pascal-ähnliche Sprache, die in einer einzelnen Datei ein volles Programm repräsentiert.
Anhand eines Beispiels werden die Grammatikregeln gezeigt, die für das Parsen der Programmiersprache benötigt werden:

\begin{figure}[H]
  \begin{lstlisting}
program foo {
    int i = 0;
    int y;
    int foo = 5;
    int[5] bar;
    int barsize = 5;
    
    foo = i + 5;
    
    for (y < barsize; y = y + 1) {
      bar[y] = x + y;
    }
    call_by_value(foo);
    call_by_reference(y);
    foo = call_with_return(y);
    write(foo);
}

func void call_by_value(int val) {
  write(val);
}

func void call_by_reference(ref int val) {
  read(val);
}

func int call_with_return(int val) {
  int x = 2;
  return x / 2 + val * 2;
}
  \end{lstlisting}
  \caption{Beispielcode in Slang}
  \label{fig:slicc:slang}
\end{figure}

Der Eintrittspunkt des Programms ist das \texttt{program}-Token, mit dessen Block die Ausführung des Programmes beginnt.
Im ersten Schritt werden dort die Variablen definiert, die im Programm verwendet werden.
Danach werden Anweisungen wie Zuweisungen, Kontrollstrukturen und Funktionsaufrufe ausgeführt.
Zwei spezielle Funktionen sind \texttt{read(i)} und \texttt{write(i)}, die standardmäßig bereitstehen und in stdin oder stdout des Programmes schreiben.
Nach dem Programmblock können Funktionen definiert werden, die im Programm selber aufgerufen werden können.\\

Eine EBNF Grammatik für die Programmiersprache Slang könnte wie folgt aussehen:

\begin{lstlisting}
/* Start der Grammatik, unser Programmcode beginnt mit program { }, */
/* dahinter stehen die Funktionsdefinitionen */
<main> ::= <program> <wso> <func_def>* <wso>

/* Ein Programm besteht aus einem Block */
<program> ::= "program" <wso> <id> <wso> <block>

<block> ::=  <wso> "{"  <wso> <block_body>  <wso> "}" <wso>

/* Ein Block besteht aus optionalen Variablendefinitionen und einer */ 
/* Anweisungsliste */
<block_body> ::= <variable_list> <wso> <statement_list>
               | <statement_list>

<variable_list> ::= <variable_declaration>
                  | <variable_declaration> <wso> <variable_list>

/* Eine Variable kann deklariert oder mit einem Wert initialisiert werden */
<variable_declaration> ::= ((<type> <wso> <id> <wso> "=" <wso> <intliteral> <wso>) | (<type> <wso> <id> <wso>)) ";"

<statement_list> ::= <statement>
                   | <statement> <wso> <statement_list>

/* Ein Statement ist eine Variablenzuweisung, eine Kontrollstruktur, */
/* eine for-Schleife, ein Funktionsaufruf oder ein `return` */
<statement> ::= <statement_assignment> <wso> ";"
              | <if_statement>
              | <for_loop>
              | <func_call> <wso> ";"
              | "return" <wso> <expression> <wso> ";"

<statement_assignment> ::= <assigment_left> <wso> "=" <wso> <expression> <wso>

/* Eine Variablenzuweisung kann auch einen Array-Eintrag referenzieren */
/* z.B. `foo[x] = 1' */
<assigment_left> ::= (<id>) | (<id> "[" <wso> (<intliteral> | <id>) <wso> "]")

/* Ein Ausdruck ist eine mathematische Operation, ein Funktionsaufruf oder ein Vergleich */
<expression> ::= <intliteral>
               | <id>
               | <func_call>
               | <expression> <wso> "+" <wso> <expression>
               | <expression> <wso> "-" <wso> <expression>
               | <expression> <wso> "*" <wso> <expression>
               | <expression> <wso> "/" <wso> <expression>
               | <compare_expression>
               | "(" <wso> <expression> <wso> ")"

<comparison_operator> ::= "=="
                        | "!="
                        | "<"
                        | "<="
                        | ">"
                        | ">="

/* Erlaube erstmal nur leichte Ausdruecke im Vergleich */
<compare_expression> ::= <id> <wso> <comparison_operator> <wso> <id>
                       | <id> <wso> <comparison_operator> <wso> <intliteral>
                       | <intliteral> <wso> <comparison_operator> <wso> <id>
                       | <intliteral> <wso> <comparison_operator> <wso> <intliteral>

/* Eine if-Kontrollstruktur mit optionalem else */
<if_statement> ::= "if" <wso> "(" <wso> <compare_expression> <wso> ")" <wso> <block>
                 | "if" <wso> "(" <wso> <compare_expression> <wso> ")"  <wso> <block> <wso> "else" <wso> <block>

/* Eine for-Schleife ohne Variablendeklaration; diese ist im ersten */
/* Teil des blocks zu definieren */
<for_loop> ::= "for" <wso> "(" <wso> <compare_expression> ";"  <wso> <statement_assignment> <wso> ")" <wso> <block>

/* Ein Funktionsaufruf mit Argumenten */
<func_call> ::= <id> <wso> "(" <wso> <func_args> <wso> ")" <wso>
<func_args> ::= <expression>
              | <expression> <wso> "," <wso> <func_args>

/* Eine Funktionsdefinition */
<func_def> ::= "func" <wso> (<type> | "void") <wso> <id> <wso> "(" <wso> <func_args_def> <wso> ")" <wso> <block>
<func_args_def> ::= <func_arg_def>
                  | <func_arg_def> <wso> "," <wso> <func_args_def>
<func_arg_def> ::= <type> <wso> <id>
                  | "ref" <wso> <type> <wso> <id> <wso>

/* Typen und Variablen */
<type> ::= "int" | <intarray>
<intarray> ::= "int[" <intliteral> "]"

/* Andere Zeichen */
<digit> ::= [0-9]
<intliteral> ::= <digit>+
<alphachar> ::= [a-z]
<id> ::= <alphachar> (<alphachar> | <digit> | "_")*
<ws> ::= " " | "\t" | "\n"

/* Optionale Leerzeichen oder newlines */
<wso> ::= <ws>*

\end{lstlisting}
\hfil\textit{Validiert mit BNF Playground\cite{paulklineBNFPlayground}}\\

\section{Parserdefinition von GNU Bison}

Während einige Parsergeneratoren mit EBNF oder BNF Grammatiken arbeiten, arbeitet Bison mit einem eigenen Definitionsformat, welches von Yacc übernommen und erweitert wurde.
Diese ist daran angelehnt, beim Parsen direkt C oder C++ Code auszuführen, der den einzelnen Grammatikregeln zugewiesen wurde.\\
Eine typische Regel in einer Bison \texttt{.y}-Datei definiert also nicht nur die Grammatikregel, sondern was mit den Werten, die aus dieser herausgelesen werden können, passieren soll:

\begin{lstlisting}
/* parser.yy */

...

%token TOK_PLUS "+"
%token TOK_MINUS "-"
%token <value> TOK_INTLITERAL

%%

expr:
  TOK_INTLITERAL { $$ = $1; }
  | expr TOK_PLUS expr { $$ = $1 + $3; }
  | expr TOK_MINUS expr { $$ = $1 - $3; }
  ;

%%
\end{lstlisting}

In diesem Beispiel wird ein einfacher Parser für mathematische Ausdrücke definiert, der die Addition oder Subtraktion zweier Integer berechnet.
Hierbei ist der Ausdruck in den geschweiften Klammern Template-Code, der beim Parsen aufgerufen wird, um aus dem Parser Daten zur Weiterverarbeitung zu bekommen.
Zum Beispiel wird hier der mathematische Ausdruck, der geparst wird, direkt ausgeführt.
Dabei bezeichnen \texttt{\$1} und \texttt{\$3} Die Werte der geparsten Tokens an der Stelle 1 und 3.
Beide lassen sich auf das Terminalsymbol \texttt{TOK\_INTLITERAL} zurückführen, welches vom Lexer bereits in eine Integer übersetzt und als \texttt{<value>} reingegeben wurde. 
Das Ergebnis der Ausführung des mathematischen Asudrucks wird in \texttt{\$\$} gespeichert, das über eine Schnittstelle auslesbar ist.\\
Die Behandlung des Syntaxbaums wird somit von Bison komplett übernommen, der Implementierer muss nur noch den Wert, der an dieser Stelle des Baumes steht, definieren.

