\chapter{Beispielcode in Slang}
\label{fig:slicc:slang}

\begin{lstlisting}
program my_calculations {
  int i = 0;
  int y;
  int foo = 5;
  int[5] bar;
  int barsize = 5;
  
  foo = i + 5;
  
  for (y < barsize; y = y + 1) {
    bar[y] = x + y;
  }
  call_by_value(foo);
  call_by_reference(y);
  if (y == 5) {
  	foo = call_with_return(y);
  }
  write(foo);
}

func void call_by_value(int val) {
	write(val);
}

func void call_by_reference(ref int val) {
	read(val);
}

func int call_with_return(int val) {
int x = 2;
return x / 2 + val * 2;
}
\end{lstlisting}

\chapter{Komplette EBNF Grammatikdefinition für Slang}
\label{fig:slicc:slang:ebnf}

\begin{lstlisting}
/* Start der Grammatik, unser Programmcode beginnt mit program { }, */
/* dahinter stehen die Funktionsdefinitionen */
<main> ::= <program> <wso> <func_def>* <wso>

/* Ein Programm besteht aus einem Block */
<program> ::= "program" <wso> <id> <wso> <block>

<block> ::=  <wso> "{"  <wso> <block_body>  <wso> "}" <wso>

/* Ein Block besteht aus optionalen Variablendefinitionen und einer */ 
/* Anweisungsliste */
<block_body> ::= <variable_list> <wso> <statement_list>
               | <statement_list>

<variable_list> ::= <variable_declaration>
                  | <variable_declaration> <wso> <variable_list>

/* Eine Variable kann deklariert oder mit einem Wert initialisiert werden */
<variable_declaration> ::= ((<type> <wso> <id> <wso> "=" <wso> <intliteral> <wso>) | (<type> <wso> <id> <wso>)) ";"

<statement_list> ::= <statement>
                   | <statement> <wso> <statement_list>

/* Ein Statement ist eine Variablenzuweisung, eine Kontrollstruktur, */
/* eine for-Schleife, ein Funktionsaufruf oder ein `return` */
<statement> ::= <statement_assignment> <wso> ";"
              | <if_statement>
              | <for_loop>
              | <func_call> <wso> ";"
              | "return" <wso> <expression> <wso> ";"

<statement_assignment> ::= <assigment_left> <wso> "=" <wso> <expression> <wso>

/* Eine Variablenzuweisung kann auch einen Array-Eintrag referenzieren */
/* z.B. `foo[x] = 1' */
<assigment_left> ::= (<id>) | (<id> "[" <wso> (<intliteral> | <id>) <wso> "]")

/* Ein Ausdruck ist eine mathematische Operation, ein Funktionsaufruf oder ein Vergleich */
<expression> ::= <intliteral>
               | <id>
               | <func_call>
               | <expression> <wso> "+" <wso> <expression>
               | <expression> <wso> "-" <wso> <expression>
               | <expression> <wso> "*" <wso> <expression>
               | <expression> <wso> "/" <wso> <expression>
               | <compare_expression>
               | "(" <wso> <expression> <wso> ")"

<comparison_operator> ::= "=="
                        | "!="
                        | "<"
                        | "<="
                        | ">"
                        | ">="

/* Erlaube erstmal nur leichte Ausdruecke im Vergleich */
<compare_expression> ::= <id> <wso> <comparison_operator> <wso> <id>
                       | <id> <wso> <comparison_operator> <wso> <intliteral>
                       | <intliteral> <wso> <comparison_operator> <wso> <id>
                       | <intliteral> <wso> <comparison_operator> <wso> <intliteral>

/* Eine if-Kontrollstruktur mit optionalem else */
<if_statement> ::= "if" <wso> "(" <wso> <compare_expression> <wso> ")" <wso> <block>
                 | "if" <wso> "(" <wso> <compare_expression> <wso> ")"  <wso> <block> <wso> "else" <wso> <block>

/* Eine for-Schleife ohne Variablendeklaration; diese ist im ersten */
/* Teil des blocks zu definieren */
<for_loop> ::= "for" <wso> "(" <wso> <compare_expression> ";"  <wso> <statement_assignment> <wso> ")" <wso> <block>

/* Ein Funktionsaufruf mit Argumenten */
<func_call> ::= <id> <wso> "(" <wso> <func_args> <wso> ")" <wso>
<func_args> ::= <expression>
              | <expression> <wso> "," <wso> <func_args>

/* Eine Funktionsdefinition */
<func_def> ::= "func" <wso> (<type> | "void") <wso> <id> <wso> "(" <wso> <func_args_def> <wso> ")" <wso> <block>
<func_args_def> ::= <func_arg_def>
                  | <func_arg_def> <wso> "," <wso> <func_args_def>
<func_arg_def> ::= <type> <wso> <id>
                  | "ref" <wso> <type> <wso> <id> <wso>

/* Typen und Variablen */
<type> ::= "int" | <intarray>
<intarray> ::= "int[" <intliteral> "]"

/* Andere Zeichen */
<digit> ::= [0-9]
<intliteral> ::= <digit>+
<alphachar> ::= [a-z]
<id> ::= <alphachar> (<alphachar> | <digit> | "_")*
<ws> ::= " " | "\t" | "\n"

/* Optionale Leerzeichen oder newlines */
<wso> ::= <ws>*

\end{lstlisting}
\hfil\textit{Validiert mit Anhang \ref{fig:slicc:slang} auf BNF Playground \cite{paulklineBNFPlayground}}\\

\chapter{SLICC Compiler Handbuch}

\section{Installation}

Der aktuelle Programmcode des Compilers ist auf GitHub verfügbar \cite{pascalernstSLICC2024}.

Um den Compiler zu bauen, wird CMake benötigt. CMake ist ein weit verbreitetes Build-System, das auf vielen Plattformen verfügbar ist.
Zusätzlich dazu wird der Kompiler \texttt{clang} benötigt sowie die Source Files zu den Bibliotheken \texttt{GNU Bison} und \texttt{Flex}.

\subsection{Linux}

Die Abhängigkeiten können auf den meisten Linuxdistributionen mit dem Paketmanager installiert werden.
Für Debian-basierte Distributionen wie Ubuntu kann dies mit folgendem Befehl getan werden:

\begin{lstlisting}
sudo apt install cmake clang bison flex
\end{lstlisting}

\subsection{MacOS}

Auf MacOS reicht es, XCode über den App Store zu installieren.
XCode kommt mit allen benötigten Tools, um den Compiler zu bauen.

\section{Bauen des Compilers}

Zum Bauen des Binärkompilates führt folgender Befehl:

\begin{lstlisting}
cmake .; make;
\end{lstlisting}

Mit dem Flag \texttt{DCMAKE\_BUILD\_TYPE=Debug} wird der Compiler im Debug-Modus gebaut:

\begin{lstlisting}
cmake -DCMAKE\_BUILD\_TYPE=Debug .; make;
\end{lstlisting}

Bei erfolgreicher Kompilierung wird ein ausführbares Programm namens \texttt{slicc} im Projektverzeichnis erstellt.

\section{Benutzung des Compilers}

Der Compiler liest den zu kompilierenden Quelltext aus Standard Input und schreibt den generierten Maschinencode in den Standard Output.

Um die Datei \texttt{examples/simple\_test.slang} mit SLICC zu kompilieren kann man diese in den Compiler pipen:

\begin{lstlisting}
cat examples/simple\_test.slang | ./slicc
\end{lstlisting}

\section{Strukturübersicht}

\begin{tikzpicture}

% Main container
\begin{umlpackage}{Hauptprogramm}
  \begin{umlpackage}[x=-3, y=0]{tac}
        \umlclass[x=0,y=0]{TacEntry}{}{}
        \umlclass[x=5,y=0]{SymbolTableEntry}{}{}
    \end{umlpackage}
\end{umlpackage}

% Frontend container
\begin{umlpackage}[x=5, y=-6]{slangparser}
    \umlclass[x=0, y=-2]{Scanner}{}{}
    \umlclass[x=0, y=2]{Parser}{}{}
    \umlclass[x=5, y=0]{FrontendDriver}{scanner : Scanner \\ parser : Parser}{parse(string)}{}
\end{umlpackage}

% Backend container
\begin{umlpackage}[x=0, y=-12]{riscgen}
    \umlclass[x=0, y=0]{BackendDriver}{generator : CodeGen}{generate(list<SymbolTableEntry>, list<TAC>) : string}{}
    \umlclass[x=7, y=0]{CodeGen}{}{}
\end{umlpackage}

% Relations
\umlunicompo{FrontendDriver}{Scanner}
\umlunicompo{FrontendDriver}{Parser}
\umlunicompo{BackendDriver}{CodeGen}
\umlunicompo{tac}{riscgen}
\umlunicompo{tac}{slangparser}

\umlassoc[geometry=-|-, arm1=2, arm2=3]{main()}{FrontendDriver.parse}
\umlassoc[geometry=|-|-, arm1=3, arm2=2]{main()}{BackendDriver.generate}

\end{tikzpicture}

Der Compiler besteht aus drei Komponentenarten: dem Frontend, dem Backend und dem Hauptprogramm.
Es können mehrere Frontends und Backends existieren, die von dem Hauptprogramm importiert und benutzt werden können.\\
Der \texttt{*Driver} in der jeweiligen Komponente ist die Klasse, die von \texttt{main()} benutzt wird, um die Funktionalität der Komponente zu nutzen.\\

Beim Kompilieren wird der Quelltext von \texttt{main()} eingelesen und an den \texttt{FrontendDriver} in der \texttt{slang\_parser} Komponente übergeben.
Diese parst den Quelltext mittels der von Flex und GNU Bison autogenerierten Klassen \texttt{Scanner} und \texttt{Parser}.
Die Zwischensprache, bestehen aus den Symboltabelleneinträgen und TACs, wird dann von \texttt{main()} an den \texttt{BackendDriver} in der \texttt{risc\_gen} Komponente übergeben.
Die Klasse \texttt{CodeGen} generiert dann den Maschinencode als String aus den Symboltabelleneinträgen und TACs und gibt sie an \texttt{main()} zurück, welche sie dann ausgibt.
