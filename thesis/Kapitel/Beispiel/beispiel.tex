\chapter{Beispiel}

\section{Das \enquote{erste} Beispiel}
Referenzen können mit \textit{\textbackslash cite\{REFERENZ\}} verwendet werden.
\cite{Martin:2008:CCH:1388398}

\subsection{Zuerst ein Unterpunkt}
\blindtext
\enquote{Der Text bekommt eine Fußnote}\footnote{Hallo herzlich Willkommen zu dieser Fußnote.}

\blindtext

\begin{figure}
	\begin{center}
		\includegraphics{Bilder/ostfalia_logo.jpg}
		\caption{Das Ostfalia Logo. Tada!}
	\end{center}
\end{figure}

\blindtext

\blindtext

\subsection{Fazit der Subsection}
\blindtext

\blindtext
\cite{Prevezanos2013}
\cite{mehta2020covid}

\blindtext

\begin{table}
	\begin{center}
		\begin{tabular}{| l c r |}
			\hline
			1 & 2 & 3 \\
			4 & 5 & 6 \\
			7 & 8 & 9 \\
			\hline
		\end{tabular}
	\end{center}
	\caption{Ist das nicht eine schöne Tabelle?}
\end{table}

\section{Das zweite Beispiel}

\subsection{Zuerst ein Unterpunkt}
\blindtext

\subsection{Fazit der Subsection}
\blindtext

\missingfigure{Hier muss noch eine Informative Grafik eingefügt werden.}

\blindtext

Hier ist bissel Text... \cite{Martin:2008:CCH:1388398}

\blindtext
\todo{Ist dieser Sachverhalt so richtig dargestellt?}

\blindtext

\subsection{Liste}
Hier wollen wir auch mal eine Liste testen:

\Blinditemize
